%///////////////////////////////////////////////////////////////////////////////
% <file type="public">
%
% <license>
%   See the README_module.xml file for this module for copyright and license
%   information.
% </license>
%   <description>
%     <abstract>
%       This is the Webnucleo Report for the libnucnet Module.
%     </abstract>
%     <keywords>
%       libnucnet Module, webnucleo report
%     </keywords>
%   </description>
%
%   <authors>
%     <current>
%       <author userid="mbradle" start_date="2007/09/29" />
%     </current>
%     <previous>
%     </previous>
%   </authors>
%
%   <compatibility>
%     TeX (Web2C 7.4.5) 3.14159 kpathsea version 3.4.5
%   </compatibility>
%
% </file>
%///////////////////////////////////////////////////////////////////////////////
%
% This is a sample LaTeX input file.  (Version of 9 April 1986)
%
% A '%' character causes TeX to ignore all remaining text on the line,
% and is used for comments like this one.

\documentclass{article}    % Specifies the document style.

\usepackage[dvips]{graphicx}
\usepackage{hyperref}

\def\prc{Phys. Rev.}
\def\araa{Ann. Rev. Astron. Astrophys.}
\def\adndt{At. Dat. Nucl. Dat. Tables}

                           % The preamble begins here.
\title{Webnucleo Technical Report: Input XML for libnucnet}  % Declares the document's title.

\author{Bradley S. Meyer}
%\date{December 12, 1984}   % Deleting this command produces today's date.

\begin{document}           % End of preamble and beginning of text.

\maketitle                 % Produces the title.


This technical report describes some details of XML input to
libnucnet.

\section{Libnucnet\_\_Nuc}  \label{sec:nuc}

The Libnucnet\_\_Nuc structure handles data about nuclei in a
collection of nuclear species.  The data for each species are the
atomic number $Z$, the mass number $A$, the atomic mass excess
$\Delta$, the ground state spin of the species, and data for the
nuclear partition function.

The nuclear mass excess is defined as
\begin{equation}
M(Z,A)c^2 = 931.478 A + \Delta(Z,A),\label{eq:mass}
\end{equation}
where $M(Z,A)c^2$ is the rest mass energy of nuclide $(Z,A)$ and the
rest mass energy and mass excess are measured in MeV.  The scaling
for the mass excess is chosen such that $\Delta(6,12)$, the mass
excess for $^{12}$C, is zero.

The nuclear partition function $G(T)$ is a sum over nuclear levels
weighted by the Boltzmann factor:
\begin{equation}
G(T) = \sum_i \left( 2 J(E_i) + 1 \right ) e^{-E_i/kT}, \label{eq:G}
\end{equation}
where the sum over $i$ is over all nuclear levels.  Because data on
the ground-state nuclear spin ($J_{g.s.}$) is already present, and
because $G(T)$ can grow rapidly with temperature, Libnucnet\_\_Nuc
actually stores the quantity $F(T)$:
\begin{equation}
F(T) = \log_{10}\left(\frac{G(T)}{2J_{g.s.} + 1}\right).
\label{eq:F}
\end{equation}

Some nuclear species have long-lived meta-stable states that must be
treated as separate nuclear species (e.g., $^{26}$Al).  These
species have their own mass excess, spin, and partition functions
(e.g., \cite{2001PhRvC..64b5805G}).
The ground state is labeled ``g'' while the
meta-stable state is labeled ``m'' (e.g., $^{26}$Al$_g$ and
$^{26}$Al$_m$).  If a nuclide has more than one meta-stable state to
be treated as a separate species, the labels are ``g'', ``m1'',
``m2'', ....

The libnucnet API allows a user to input nuclear data directly;
however, the most convenient method to read in the relevant
information is via an XML file.  The schema that defines the grammar
for such an input file is libnucnet\_\_nuc.xsd in the xsd\_pub
for a version of libnucnet.

An example of an input XML file is the following, which contains
data for $^1H$ and $^{12}$C:
\begin{verbatim}
<?xml version="1.0" encoding="UTF-8"?>

<nuclear_data>

  <!--h1-->

  <nuclide>
    <z>1</z>
    <a>1</a>
    <source>Tuli (2000)</source>
    <mass_excess>7.289</mass_excess>
    <spin>0.5</spin>
    <partf_table>
      <point>
        <t9>0.1</t9>
        <log10_partf>0</log10_partf>
      </point>
      <point>
        <t9>0.15</t9>
        <log10_partf>0</log10_partf>
      </point>
      <point>
        <t9>0.2</t9>
        <log10_partf>0</log10_partf>
      </point>
      <point>
        <t9>0.3</t9>
        <log10_partf>0</log10_partf>
      </point>
      <point>
        <t9>0.4</t9>
        <log10_partf>0</log10_partf>
      </point>
      <point>
        <t9>0.5</t9>
        <log10_partf>0</log10_partf>
      </point>
      <point>
        <t9>0.9</t9>
        <log10_partf>0</log10_partf>
      </point>
      <point>
        <t9>1</t9>
        <log10_partf>0</log10_partf>
      </point>
      <point>
        <t9>1.5</t9>
        <log10_partf>0</log10_partf>
      </point>
      <point>
        <t9>2</t9>
        <log10_partf>0</log10_partf>
      </point>
      <point>
        <t9>2.5</t9>
        <log10_partf>0</log10_partf>
      </point>
      <point>
        <t9>3</t9>
        <log10_partf>0</log10_partf>
      </point>
      <point>
        <t9>3.5</t9>
        <log10_partf>0</log10_partf>
      </point>
      <point>
        <t9>4</t9>
        <log10_partf>0</log10_partf>
      </point>
      <point>
        <t9>5</t9>
        <log10_partf>0</log10_partf>
      </point>
      <point>
        <t9>10</t9>
        <log10_partf>0</log10_partf>
      </point>
    </partf_table>
  </nuclide>

  <!--c12-->

  <nuclide>
    <z>6</z>
    <a>12</a>
    <source>Tuli (2000)</source>
    <mass_excess>0.</mass_excess>
    <spin>0.</spin>
    <partf_table>
      <point>
        <t9>0.1</t9>
        <log10_partf>0</log10_partf>
      </point>
      <point>
        <t9>0.15</t9>
        <log10_partf>0</log10_partf>
      </point>
      <point>
        <t9>0.2</t9>
        <log10_partf>0</log10_partf>
      </point>
      <point>
        <t9>0.3</t9>
        <log10_partf>0</log10_partf>
      </point>
      <point>
        <t9>0.4</t9>
        <log10_partf>0</log10_partf>
      </point>
      <point>
        <t9>0.5</t9>
        <log10_partf>0</log10_partf>
      </point>
      <point>
        <t9>0.9</t9>
        <log10_partf>0</log10_partf>
      </point>
      <point>
        <t9>1</t9>
        <log10_partf>0</log10_partf>
      </point>
      <point>
        <t9>1.5</t9>
        <log10_partf>0</log10_partf>
      </point>
      <point>
        <t9>2</t9>
        <log10_partf>0</log10_partf>
      </point>
      <point>
        <t9>2.5</t9>
        <log10_partf>0</log10_partf>
      </point>
      <point>
        <t9>3</t9>
        <log10_partf>0</log10_partf>
      </point>
      <point>
        <t9>3.5</t9>
        <log10_partf>0</log10_partf>
      </point>
      <point>
        <t9>4</t9>
        <log10_partf>0</log10_partf>
      </point>
      <point>
        <t9>5</t9>
        <log10_partf>0</log10_partf>
      </point>
      <point>
        <t9>10</t9>
        <log10_partf>0</log10_partf>
      </point>
    </partf_table>
  </nuclide>

</nuclear_data>

\end{verbatim}

In this file, the root tag is {\bf nuclear\_data}.  The data are a
sequence of nuclides ({\bf nuclide} tag).  The lines like
\begin{verbatim}
<!-- h1 -->
\end{verbatim}
are optional comments that are ignored by the XML parser. There is
no bound of the number of nuclides that may be present in the file.
Data for a nuclide are a sequence of {\bf z}, {\bf a}, {\bf source},
{\bf mass\_excess}, {\bf spin}, and {\bf partf\_table} tags. The data for
these tags are:
\begin{itemize}

\item[{\bf z:}] The nuclide's atomic number (required).  It must be a
non-negative integer.

\item[{\bf a:}] The nuclide's mass number (required).  It must be a
positive integer.

\item[{\bf source:}] A string giving the data source for the file (usually
a reference to a paper from which the data are taken).  This tag is
optional.

\item[{\bf mass\_excess:}] The nuclide's mass excess in MeV (see Eq.
(\ref{eq:mass}).  This is a floating point number, and the tag is
required.  Note that this tag was changed from {\bf mass}
to {\bf mass\_excess} as of version 0.2 of libnucnet.

\item[{\bf spin:}] The nuclide's spin (required).  This is a non-negative
floating point number.

\end{itemize}

The partition function data are those contained between the {\bf
partf\_table} tags.  The data are to be thought of as a table with
each table entry between the {\bf point} tags.  The data for each
table point are:
\begin{itemize}

\item[{\bf t9:}] The temperature in billions of Kelvins (required).  It
must be a non-negative float.

\item[{\bf log10\_partf:}]  The $F(T)$ factor [see Eq. (\ref{eq:F})]
evaluated at the $T_9$ of this point (required).  It is a float.

\end{itemize}
As of version 0.4, it is no longer necessary for the partition function
data to be sorted
in ascending order by temperature.  The routine to update the input data sorts
them before storing.

The libnucnet API routine that computes the nuclear partition
function for a nuclide interpolates the $F(T)$ from the values in
the table using the GNU Scientific Library spline interpolation
routine.  It then computes the nuclear partition function as
\[
G(T) = \left( 2 J_{g.s.} + 1 \right ) 10^{F(T)}.
\]
If the input temperature is less than the first {\bf t9} of the
table, the routine simply uses the lowest temperature point's $F$.
Similarly, if the input temperature is greater than the last {\bf
t9} of the table, the routine uses the highest temperature point's
$F$.  In other words, the partition function routine does not
extrapolate.  The user should therefore supply the partition
function table data for the fully desired temperature range.
If only two partition function data points are provided, spline interpolation
does not work.  Instead, as of version 0.4, the routine interpolates
linearly between the two points.

The routine that parses in the nuclear data will replace data for
a given nuclide.  Thus, the input data may contain multiple entries
for a given nuclide.  Suppose, for example, the input XML file has
multiple nuclear data entries for $^{28}$Si.  The output from the parsing
routine Libnucnet\_\_Nuc\_\_new\_from\_xml() will use the last entry
(i.e., closest to the end of the file) for $^{28}$Si.  The user can thus
add data to the end of the nuclear data file (but within the
{\bf nuclear\_data} tags) to replace the original data for a nuclide
with his or her own; however, a better solution is to use the API routine
Libnucnet\_\_Nuc\_\_updateFromXml(), as demonstrated in the Libnucnet\_\_Nuc
Examples Tutorial.

When a nuclide has multiple states, as discussed above, the data for
the mass excess, spin, and partition function are contained in the
{\bf states} tags.  The data for each state is between {\bf state}
tags.  The {\bf id} tag is label for the state (``g" for the ground
state; ``m" for a single meta-stable state; ``m1", ``m2", ... for
multiple meta-stable states.  An example of XML data for a nuclide
with multiple states is the following:

\begin{verbatim}

 <!--al26-->
  <nuclide>
    <z>13</z>
    <a>26</a>
    <states>
    <state id="g">
      <source>Tuli (2000) + Gupta and Meyer (2001)</source>
    <mass_excess>-12.21</mass_excess>
    <spin>5</spin>
    <partf_table>
      <point>
        <t9> 0.0100</t9>
        <log10_partf>0.000000</log10_partf>
      </point>
      <point>
        <t9> 0.1000</t9>
        <log10_partf>0.000000</log10_partf>
      </point>
      <point>
        <t9> 0.2000</t9>
        <log10_partf>0.000000</log10_partf>
      </point>
      <point>
        <t9> 0.3000</t9>
        <log10_partf>0.000000</log10_partf>
      </point>
      <point>
        <t9> 0.4000</t9>
        <log10_partf>0.000000</log10_partf>
      </point>
      <point>
        <t9> 0.5000</t9>
        <log10_partf>0.000017</log10_partf>
      </point>
      <point>
        <t9> 0.6000</t9>
        <log10_partf>0.000087</log10_partf>
      </point>
      <point>
        <t9> 0.7000</t9>
        <log10_partf>0.000274</log10_partf>
      </point>
      <point>
        <t9> 0.8000</t9>
        <log10_partf>0.000651</log10_partf>
      </point>
      <point>
        <t9> 0.9000</t9>
        <log10_partf>0.001279</log10_partf>
      </point>
      <point>
        <t9> 1.0000</t9>
        <log10_partf>0.002183</log10_partf>
      </point>
      <point>
        <t9> 2.0000</t9>
        <log10_partf>0.013907</log10_partf>
      </point>
      <point>
        <t9> 3.0000</t9>
        <log10_partf>0.005404</log10_partf>
      </point>
      <point>
        <t9> 4.0000</t9>
        <log10_partf>0.003288</log10_partf>
      </point>
      <point>
        <t9> 5.0000</t9>
        <log10_partf>0.003779</log10_partf>
      </point>
      <point>
        <t9> 6.0000</t9>
        <log10_partf>0.007458</log10_partf>
      </point>
      <point>
        <t9> 7.0000</t9>
        <log10_partf>0.015913</log10_partf>
      </point>
      <point>
        <t9> 8.0000</t9>
        <log10_partf>0.030094</log10_partf>
      </point>
      <point>
        <t9> 9.0000</t9>
        <log10_partf>0.049427</log10_partf>
      </point>
      <point>
        <t9>10.0000</t9>
        <log10_partf>0.072309</log10_partf>
      </point>
    </partf_table>
    </state>
    <state id="m">
      <source>Tuli (2000) + Gupta and Meyer (2001)</source>
    <mass_excess>-11.982</mass_excess>
    <spin>0</spin>
    <partf_table>
      <point>
        <t9> 0.0100</t9>
        <log10_partf>0.000000</log10_partf>
      </point>
      <point>
        <t9> 0.1000</t9>
        <log10_partf>0.000000</log10_partf>
      </point>
      <point>
        <t9> 0.2000</t9>
        <log10_partf>0.000000</log10_partf>
      </point>
      <point>
        <t9> 0.2500</t9>
        <log10_partf>0.000000</log10_partf>
      </point>
      <point>
        <t9> 0.3000</t9>
        <log10_partf>0.000000</log10_partf>
      </point>
      <point>
        <t9> 0.4000</t9>
        <log10_partf>0.000000</log10_partf>
      </point>
      <point>
        <t9> 0.5000</t9>
        <log10_partf>0.000000</log10_partf>
      </point>
      <point>
        <t9> 0.6000</t9>
        <log10_partf>0.000000</log10_partf>
      </point>
      <point>
        <t9> 0.7000</t9>
        <log10_partf>0.000004</log10_partf>
      </point>
      <point>
        <t9> 0.8000</t9>
        <log10_partf>0.000013</log10_partf>
      </point>
      <point>
        <t9> 0.9000</t9>
        <log10_partf>0.000056</log10_partf>
      </point>
      <point>
        <t9> 1.0000</t9>
        <log10_partf>0.000187</log10_partf>
      </point>
      <point>
        <t9> 2.0000</t9>
        <log10_partf>0.306116</log10_partf>
      </point>
      <point>
        <t9> 3.0000</t9>
        <log10_partf>0.623279</log10_partf>
      </point>
      <point>
        <t9> 4.0000</t9>
        <log10_partf>0.729541</log10_partf>
      </point>
      <point>
        <t9> 5.0000</t9>
        <log10_partf>0.806769</log10_partf>
      </point>
      <point>
        <t9> 6.0000</t9>
        <log10_partf>0.875325</log10_partf>
      </point>
      <point>
        <t9> 7.0000</t9>
        <log10_partf>0.939000</log10_partf>
      </point>
      <point>
        <t9> 8.0000</t9>
        <log10_partf>0.999050</log10_partf>
      </point>
      <point>
        <t9> 9.0000</t9>
        <log10_partf>1.057029</log10_partf>
      </point>
      <point>
        <t9>10.0000</t9>
        <log10_partf>1.114257</log10_partf>
      </point>
    </partf_table>
    </state>
    </states>
  </nuclide>

\end{verbatim}

\section{Libnucnet\_\_Reac}  \label{sec:reac}

The Libnucnet\_\_Reac structure handles data about a collection of
nuclear reactions. The data for each reaction are the source of the
data, the names of the reactants and products, and the reaction rate
data.  The rate data are in the form of a single rate, a rate table,
or a NON SMOKER fit.  An example of a libnucnet reaction file is the
following:

\begin{verbatim}

<?xml version="1.0" encoding="ISO-8859-1"?>

<reaction_data>

<!-- h1 + n to h2 + gamma -->

     <reaction>
       <source>Smith et al. (1993)</source>
       <reactant>h1</reactant><reactant>n</reactant>
       <product>h2</product><product>gamma</product>
       <rate_table>
         <point>
           <t9>0.001</t9>
           <rate>4.6168E+04</rate>
           <sef>1.000</sef>
         </point>
         <point>
           <t9>0.002</t9>
           <rate>4.5663E+04</rate>
           <sef>1.000</sef>
         </point>
         <point>
           <t9>0.003</t9>
           <rate>4.5281E+04</rate>
           <sef>1.000</sef>
         </point>
         <point>
           <t9>0.004</t9>
           <rate>4.4963E+04</rate>
           <sef>1.000</sef>
         </point>
         <point>
           <t9>0.005</t9>
           <rate>4.4684E+04</rate>
           <sef>1.000</sef>
         </point>
         <point>
           <t9>0.006</t9>
           <rate>4.4435E+04</rate>
           <sef>1.000</sef>
         </point>
         <point>
           <t9>0.007</t9>
           <rate>4.4208E+04</rate>
           <sef>1.000</sef>
         </point>
         <point>
           <t9>0.008</t9>
           <rate>4.3997E+04</rate>
           <sef>1.000</sef>
         </point>
         <point>
           <t9>0.009</t9>
           <rate>4.3801E+04</rate>
           <sef>1.000</sef>
         </point>
         <point>
           <t9>0.010</t9>
           <rate>4.3617E+04</rate>
           <sef>1.000</sef>
         </point>
         <point>
           <t9>0.020</t9>
           <rate>4.2172E+04</rate>
           <sef>1.000</sef>
         </point>
         <point>
           <t9>0.030</t9>
           <rate>4.1112E+04</rate>
           <sef>1.000</sef>
         </point>
         <point>
           <t9>0.040</t9>
           <rate>4.0252E+04</rate>
           <sef>1.000</sef>
         </point>
         <point>
           <t9>0.050</t9>
           <rate>3.9519E+04</rate>
           <sef>1.000</sef>
         </point>
         <point>
           <t9>0.060</t9>
           <rate>3.8875E+04</rate>
           <sef>1.000</sef>
         </point>
         <point>
           <t9>0.070</t9>
           <rate>3.8300E+04</rate>
           <sef>1.000</sef>
         </point>
         <point>
           <t9>0.080</t9>
           <rate>3.7778E+04</rate>
           <sef>1.000</sef>
         </point>
         <point>
           <t9>0.090</t9>
           <rate>3.7299E+04</rate>
           <sef>1.000</sef>
         </point>
         <point>
           <t9>0.100</t9>
           <rate>3.6857E+04</rate>
           <sef>1.000</sef>
         </point>
         <point>
           <t9>0.200</t9>
           <rate>3.3649E+04</rate>
           <sef>1.000</sef>
         </point>
         <point>
           <t9>0.300</t9>
           <rate>3.1600E+04</rate>
           <sef>1.000</sef>
         </point>
         <point>
           <t9>0.400</t9>
           <rate>3.0131E+04</rate>
           <sef>1.000</sef>
         </point>
         <point>
           <t9>0.500</t9>
           <rate>2.9022E+04</rate>
           <sef>1.000</sef>
         </point>
         <point>
           <t9>0.600</t9>
           <rate>2.8161E+04</rate>
           <sef>1.000</sef>
         </point>
         <point>
           <t9>0.700</t9>
           <rate>2.7481E+04</rate>
           <sef>1.000</sef>
         </point>
         <point>
           <t9>0.800</t9>
           <rate>2.6942E+04</rate>
           <sef>1.000</sef>
         </point>
         <point>
           <t9>0.900</t9>
           <rate>2.6514E+04</rate>
           <sef>1.000</sef>
         </point>
         <point>
           <t9>1.000</t9>
           <rate>2.6175E+04</rate>
           <sef>1.000</sef>
         </point>
         <point>
           <t9>2.000</t9>
           <rate>2.5518E+04</rate>
           <sef>1.000</sef>
         </point>
         <point>
           <t9>3.000</t9>
           <rate>2.7018E+04</rate>
           <sef>1.000</sef>
         </point>
         <point>
           <t9>4.000</t9>
           <rate>2.9256E+04</rate>
           <sef>1.000</sef>
         </point>
         <point>
           <t9>5.000</t9>
           <rate>3.1766E+04</rate>
           <sef>1.000</sef>
         </point>
         <point>
           <t9>6.000</t9>
           <rate>3.4347E+04</rate>
           <sef>1.000</sef>
         </point>
         <point>
           <t9>7.000</t9>
           <rate>3.6902E+04</rate>
           <sef>1.000</sef>
         </point>
         <point>
           <t9>8.000</t9>
           <rate>3.9381E+04</rate>
           <sef>1.000</sef>
         </point>
         <point>
           <t9>9.000</t9>
           <rate>4.1759E+04</rate>
           <sef>1.000</sef>
         </point>
         <point>
           <t9>10.000</t9>
           <rate>4.4025E+04</rate>
           <sef>1.000</sef>
         </point>
       </rate_table>
     </reaction>

<!-- ne15 + n to ne16 + gamma -->

<reaction>
   <source>ADNDT (2001) 75, 1 (non-smoker)</source>
   <reactant>ne15</reactant>  <reactant>n</reactant>
   <product>ne16</product>  <product>gamma</product>
   <non_smoker_fit>
      <Zt> 10</Zt>
      <At> 15</At>
      <Zf> 10</Zf>
      <Af> 16</Af>
      <Q>     8.071000</Q>
      <spint> 0.0000</spint>
      <spinf> 0.0000</spinf>
      <TlowHf>-1.0000</TlowHf>
      <Tlowfit> 0.0100</Tlowfit>
      <Thighfit> 10.</Thighfit>
      <acc> 1.900000e-06</acc>
      <a1> 6.225343e+00</a1>
      <a2> 1.023384e-02</a2>
      <a3>-1.272184e+00</a3>
      <a4> 3.920127e+00</a4>
      <a5>-1.966720e-01</a5>
      <a6> 1.394263e-02</a6>
      <a7>-1.389816e+00</a7>
      <a8> 2.983430e+01</a8>
   </non_smoker_fit>
</reaction>

<!-- o15 to n15 + positron + neutrino_e -->

<reaction>
   <source>Tuli (2000)</source>
   <reactant>o15</reactant>
   <product>n15</product>  <product>positron</product>
   <product>neutrino_e</product>
   <single_rate>5.6704e-3</single_rate>
</reaction>

<!--ne24 + positron -> na24 + anti-neutrino_e-->
<reaction>
  <source>ffn rates</source>
  <reactant>ne24</reactant>
  <reactant>positron</reactant>
  <product>na24</product>
  <product>anti-neutrino_e</product>
  <user_rate key="two-d weak rates">
    <properties>
      <property name="Extra information">
        Data are from Fuller, Fowler, and Newman (1982).  The user
        interpolates the 2-d table given input t9 and electron density
        (Ye * mass density).
      </property>
      <property name="log10_rhoe" tag1="0">1.000</property>
      <property name="log10_rhoe" tag1="1">2.000</property>
      <property name="log10_rhoe" tag1="2">3.000</property>
      <property name="log10_rhoe" tag1="3">4.000</property>
      <property name="log10_rhoe" tag1="4">5.000</property>
      <property name="log10_rhoe" tag1="5">6.000</property>
      <property name="log10_rate" tag1="0" tag2="0">-.1000E+03</property>
      <property name="log10_rate" tag1="0" tag2="1">-.1000E+03</property>
      <property name="log10_rate" tag1="0" tag2="2">-.1000E+03</property>
      <property name="log10_rate" tag1="0" tag2="3">-.1000E+03</property>
      <property name="log10_rate" tag1="0" tag2="4">-.1000E+03</property>
      <property name="log10_rate" tag1="0" tag2="5">-.1000E+03</property>
      <property name="log10_rate" tag1="1" tag2="0">-.5453E+02</property>
      <property name="log10_rate" tag1="1" tag2="1">-.5553E+02</property>
      <property name="log10_rate" tag1="1" tag2="2">-.5655E+02</property>
      <property name="log10_rate" tag1="1" tag2="3">-.5770E+02</property>
      <property name="log10_rate" tag1="1" tag2="4">-.6005E+02</property>
      <property name="log10_rate" tag1="1" tag2="5">-.6824E+02</property>
      <property name="log10_rate" tag1="2" tag2="0">-.2767E+02</property>
      <property name="log10_rate" tag1="2" tag2="1">-.2867E+02</property>
      <property name="log10_rate" tag1="2" tag2="2">-.2968E+02</property>
      <property name="log10_rate" tag1="2" tag2="3">-.3073E+02</property>
      <property name="log10_rate" tag1="2" tag2="4">-.3224E+02</property>
      <property name="log10_rate" tag1="2" tag2="5">-.3642E+02</property>
      <property name="log10_rate" tag1="3" tag2="0">-.1371E+02</property>
      <property name="log10_rate" tag1="3" tag2="1">-.1471E+02</property>
      <property name="log10_rate" tag1="3" tag2="2">-.1571E+02</property>
      <property name="log10_rate" tag1="3" tag2="3">-.1673E+02</property>
      <property name="log10_rate" tag1="3" tag2="4">-.1790E+02</property>
      <property name="log10_rate" tag1="3" tag2="5">-.2017E+02</property>
      <property name="log10_rate" tag1="4" tag2="0">-.7939E+01</property>
      <property name="log10_rate" tag1="4" tag2="1">-.8352E+01</property>
      <property name="log10_rate" tag1="4" tag2="2">-.9301E+01</property>
      <property name="log10_rate" tag1="4" tag2="3">-.1031E+02</property>
      <property name="log10_rate" tag1="4" tag2="4">-.1137E+02</property>
      <property name="log10_rate" tag1="4" tag2="5">-.1292E+02</property>
      <property name="log10_rate" tag1="5" tag2="0">-.6462E+01</property>
      <property name="log10_rate" tag1="5" tag2="1">-.6483E+01</property>
      <property name="log10_rate" tag1="5" tag2="2">-.6689E+01</property>
      <property name="log10_rate" tag1="5" tag2="3">-.7511E+01</property>
      <property name="log10_rate" tag1="5" tag2="4">-.8542E+01</property>
      <property name="log10_rate" tag1="5" tag2="5">-.9846E+01</property>
      <property name="t9" tag1="0">.010</property>
      <property name="t9" tag1="1">.100</property>
      <property name="t9" tag1="2">.200</property>
      <property name="t9" tag1="3">.400</property>
      <property name="t9" tag1="4">.700</property>
      <property name="t9" tag1="5">1.000</property>
    </properties>
  </user_rate>
</reaction>

</reaction_data>

\end{verbatim}

The root tag for the file is {\bf reaction\_data}.  The data are a
sequence of reactions ({\bf reaction} tags).  There is no bound on
the number of reactions that can be present in the file.  For each
reaction, the following tags are present:

\begin{itemize}

\item[{\bf source:}] A string giving the reaction source for the
data.  This is an optional tag, and the string is usually a
reference to a publication with the data.

\item[{\bf reactant:}] A string giving the name of one of the
reactants (required).  The number of reactants that may be present
is unbounded, but there needs to be at least one reactant.

\item[{\bf product:}] A string giving the name of one of the products
(required).  The number of products that may be present is
unbounded, but there needs to be at least one product.

\end{itemize}

The reaction string is constructed from the reactants and products.
The nuclear species are named with small letters, a number, and, if
present, the state label.  For example, $^{16}$O is o16 as a
reactant or product. The reaction string constructed from the
reactants and products should be valid, that is, it should satisfy
baryon number, lepton number, and charge conservation.  To do this,
the reactants or products are not just nuclear species but can also
be ``gamma" (for gamma-rays), ``electron" (for an electron, that is,
an $e^-$), ``positron" (for a positron, that is, an $e^+$), and
``neutrino\_e" (for an electron-type neutrino, that is, a $\nu_e$),
``anti-neutrino\_e" (for an electron-type anti-neutrino, that is, a
${\bar \nu}_e$).  Other allowed leptons are muons (``mu''), anti-muons
(``anti-mu''), tauons (``tau''), anti-tauons (``anti-tau''), and their
corresponding neutrinos (``neutrino\_mu'', ``anti-neutrino\_mu'',
``neutrino\_tau'', and ``anti-neutrin\_tau'').

The rate data follow the reactants and products.  At present, the
rate data can be one of four types.  Only one type can be present
for a particular reaction.  For a rate that is the same at all
temperatures, the data are between {\bf single\_rate} tags:

\begin{itemize}

\item[{\bf single\_rate:}] The rate per nuclide per second.  This is
the data for a rate that is the same at all temperatures.

\end{itemize}

For a rate whose data are contained in a table ({\bf rate\_table}
tag), the data are in table entries ({\bf points}):

\begin{itemize}

\item[{\bf t9:}] The temperature in billions of Kelvins (required).
This is a non-negative float.

\item[{\bf rate:}] The rate per interaction pair or triplet (or
higher multiplet) per second (required).  This is a non-negative
float.

\item[{\bf sef:}] The stellar enhancement factor (required).  This is
a non-negative float.  If the {\bf rate} datum is for the ground state,
libnucnet multiplies the rate by the sef to correct for the excited
state rate.  If the rate for the given temperature is already
corrected for excited states, an sef of 1.0 should be used.

\end{itemize}
As of version 0.4, it is no longer necessary for the table data to be sorted
in ascending order by temperature.  The routine to update the input data sorts
them before storing.

libnucnet computes rates from rate table data by interpolating the
$\log_{10}$ of the product of the rate and the sef with the GNU
Scientific Library spline interpolation routines.  The result is
then exponentiated.  libnucnet does not extrapolate beyond the
table.  If the input temperature is lower than the lowest
temperature in the table, the lowest temperature value for the rate
times sef is used.  If the input temperature is higher than the
highest temperature in the table, the highest temperature value for
the rate times sef is used.  The user should therefore supply rate
data for the full temperature range expected in the problem.
If only two points are provided for the table, the rate is computed between
those points by linear interpolation, not by the spline routines.

For a rate that is given by a non-smoker fit ({\bf
non\_smoker\_fit}) \cite{2000ADNDT..75....1R},
the data are between the following tags:

\begin{itemize}

\item[{\bf Zt:}]  Atomic number of the target nucleus (optional).
It is a non-negative integer.
\item[{\bf At:}]  Atomic mass of the target nucleus (optional).  It
is a positive integer.
\item[{\bf Zf:}]  Atomic number of the final nucleus (optional).  It
is a non-negative integer.
\item[{\bf Af:}]  Atomic mass of the final nucleus (optional).  It
is a positive integer.
\item[{\bf Q:}]  Q value of the reaction (optional).  It is a float.
libnucnet does not use this but rather calculates it from data from
the nuclear data file.
\item[{\bf spint:}]  Spin of the target nucleus (optional).  It is a non-negative float.
libnucnet does not use this but rather uses data from the nuclear
data file.
\item[{\bf spinf:}] Spin of the final nucleus (optional).  It is a non-negative float.
libnucnet does not use this but rather uses data from the nuclear
data file.
\item[{\bf TlowHf:}]  Lowest value for which the Hausher-Feshbach
fit works (optional).  It is a float.
\item[{\bf Tlowfit:}]  Lowest value for which the
fit works (required).  It is a float.
\item[{\bf Thighfit:}]  Highest value for which the
fit works (optional).  If not supplied, it is assumed to be a default value,
initially set to 10.  It is a float.
\item[{\bf acc:}]  Accuracy of the fit (optional).  It is a non-zero
float.
\item[{\bf a1:}]  First non-smoker fit parameter (required).  It is a
float.
\item[{\bf a2:}]  Second non-smoker fit parameter (required).  It is
a float.
\item[{\bf a3:}]  Third non-smoker fit parameter (required).  It is
a float.
\item[{\bf a4:}]  Fourth non-smoker fit parameter (required).  It is
a float.
\item[{\bf a5:}]  Fifth non-smoker fit parameter (required).  It is
a float.
\item[{\bf a6:}]  Sixth non-smoker fit parameter (required).  It is
a float.
\item[{\bf a7:}]  Seventh non-smoker fit parameter (required).  It
is a float.
\item[{\bf a8:}]  Eighth non-smoker fit parameter (optional).  It is
a float.  libnucnet does not actually use this parameter.

\end{itemize}

The rate per interacting pair, triplet, or higher multiplet is
computed by the Non-Smoker fit as

\begin{equation}
exp\left( a1 + a2/T_9 + a3 / T_9^{1/3} + a4 T_9^{1/3} + a5 T_9 + a6
T_9^{5/3} + a7 \log T_9  \right ) \label{eq:nonSmokerFit}
\end{equation}

As of version 0.2, if the input temperature in billions of K is less than {\bf
Tlowfit}, libnucnet computes the reaction rate at {\bf Tlowfit}.  Version
0.1 of libnucnet set the rate to zero for an input temperature less than
{\bf Tlowfit}.  As of version 0.3, if the temperature in billions of K is
greater than {\bf Thighfit}, libnucnet computes the reaction rates at
{\bf Thighfit}.  If the user does not supply {\bf Thighfit}, it is taken as a
default value, which is set by D\_THIGHTFIT\_DEFAULT in Libnucnet\_\_Reac.h
and, in the distribution, is set to 10.

Also as of version 0.2, libnucnet is able to handle multiple non-smoker fits.
Here the format is similar to the non-smoker fit data above except that
there are now {\bf fit} tags between the {\bf non\_smoker\_fit} tags.
An example is the following:

\begin{verbatim}

  <!--he4 + he4 + he4 -> c12 + gamma-->
  <reaction>
    <source>non smoker example</source>
    <reactant>he4</reactant>
    <reactant>he4</reactant>
    <reactant>he4</reactant>
    <product>c12</product>
    <product>gamma</product>
    <non_smoker_fit>
      <fit note="non-resonant part">
        <spint>0</spint>
        <spinf>0</spinf>
        <TlowHf>-1</TlowHf>
        <Tlowfit>0.01</Tlowfit>
        <Thighfit>12.</Tlowfit>
        <acc>1.9e-06</acc>
        <a1>5.3463</a1>
        <a2>0</a2>
        <a3>-37.1289</a3>
        <a4>14.2705</a4>
        <a5>-92.885</a5>
        <a6>-20.4254</a6>
        <a7>-0.666667</a7>
        <a8>0</a8>
      </fit>
      <fit note="The first resonance part">
        <spint>0</spint>
        <spinf>0</spinf>
        <TlowHf>-1</TlowHf>
        <Tlowfit>0.01</Tlowfit>
        <Thighfit>10.</Tlowfit>
        <acc>1.9e-06</acc>
        <a1>-11.8694</a1>
        <a2>-4.32998</a2>
        <a3>0</a3>
        <a4>-6.24062</a4>
        <a5>0.715957</a5>
        <a6>-0.0561058</a6>
        <a7>-1.5</a7>
        <a8>0</a8>
      </fit>
      <fit note="The second resonance part">
        <spint>0</spint>
        <spinf>0</spinf>
        <TlowHf>-1</TlowHf>
        <Tlowfit>0.01</Tlowfit>
        <Thighfit>8.</Tlowfit>
        <acc>1.9e-06</acc>
        <a1>-121.677</a1>
        <a2>-1.36658</a2>
        <a3>0</a3>
        <a4>1.86071</a4>
        <a5>-130.231</a5>
        <a6>-7.77528</a6>
        <a7>-1.5</a7>
        <a8>0</a8>
      </fit>
    </non_smoker_fit>
  </reaction>

\end{verbatim}

Each fit has the same form as a single non-smoker fit.  The optional
note attribute allows the user the identify the different fits.  The total
reaction rate is the sum of the rate from each fit, as computed by
Eq. (\ref{eq:nonSmokerFit}).

As of version 0.5, it is possible for users to supply their own function
to be applied during calculation of a reaction rate.  The data for the
function are supplied between {\bf user\_rate} tags, which have a required
{\bf key} attribute that identifies the user-supplied function associated
with the data.  The data for the function are ``properties'', so the data
are enclosed between {\bf properties} tags and each datum is enclosed in {\bf
property} tags, which is a string representing any kind of data type.
Each {\bf property} has the following attributes:

\begin{itemize}

\item[{\bf name:}]  The name associated with the property (required).
It is a string.
\item[{\bf tag1:}]  A tag associated with the property (optional).  It
is a string and helps distinguish properties with the same name.
\item[{\bf tag2:}]  Another tag associated with the property (optional).  It
is a string and helps distinguish properties with the same name and tag1.

\end{itemize}

A libnucnet technical report describes application of user-supplied rate
functions in more detail, and examples in the libnucnet distribution
demonstrate their use.

\section{Libnucnet\_\_Net}

The Libnucnet\_\_Net structure consists of a collection of
nuclei and reactions among them.  The input is the nuclear
data and the reaction data, and the XML file is therefore simply
a combination of the nuclear data and the reaction data.  It has the
form then

\begin{verbatim}

<?xml version="1.0" encoding="ISO-8859-1"?>

<nuclear_network>

  <nuclear_data>
  .
  .
  .
  </nuclear_data>

  <reaction_data>
  .
  .
  .
  </reaction_data>

</nuclear_network>

\end{verbatim}

An xslt stylesheet provided with the libnucnet distribution allows
the user to combine a nuclear data and reaction data input XML file
to construct a Libnucnet\_\_Net input XML file easily.

A valid reaction is defined as one that satisfies conservation of
baryon number, lepton number, and charge and occurs
between nuclei included in the network (that is, nuclei that
have data in the
{\bf nuclear\_data} part of the input file).  Routines attached to
Libnucnet\_\_Net compute reverse reaction rates by detailed balance.
To compute reverse rates, libnucnet uses the nuclear data  from the
{\bf nuclear\_data} part of the Libnucnet\_\_Net input XML file.  In
particular, the reverse reaction rate routines use the $Q$ value of
the reaction computed from the mass excesses and nuclear partition
functions of the reactants and products.  The reverse rate is computed
by detailed balance
as described, for example, in \cite{1967ARA&A...5..525F}.
Clearly data for all
reactants and products in a reaction must be present for the reverse
rate to be calculated.  Thus libnucnet only computes reverse rates
for valid reactions.

\section{Zone Data}

The time-dependent data in libnucnet is contained in zones.  Data for each
zone can be read in from an input XML.  As of version 0.3, the XML data for
zones is placed between {\bf zone\_data} tags.  For example,
for a single-zone calculation, the
input XML file would look like:

\begin{verbatim}

<zone_data>

  <zone>

    <mass_fractions>
      <nuclide name="n">
        <z>0</z>
        <a>1</a>
        <x>0.5</x>
      </nuclide>
      <nuclide>
        <z>1</z>
        <a>1</a>
        <x>0.3</x>
      </nuclide>
      <nuclide>
        <z>6</z>
        <a>12</a>
        <x>0.2</x>
      </nuclide>
    </mass_fractions>

  </zone>

</zone_data>

\end{verbatim}

The data between the {\bf zone} tags are for a set of {\bf mass\_fractions},
which consist of a sequence of
nuclides ({\bf nuclide} tag).  For each nuclide, the data are:
\begin{itemize}
\item[{\bf name:}] The name of the nuclide as an attribute (optional).
\item[{\bf z:}] The atomic number of the nuclide (required).  This
is a non-negative integer.
\item[{\bf a:}] The mass number of the nuclide (required).  This is
a positive integer.
\item[{\bf x:}] The mass fraction of the species (required).  This
is a float whose value lies between 0 and 1.
\end{itemize}
The sum of all the mass fractions for all the nuclides should be
unity, although this is not required.  If both the nuclide {\bf name} and the
{\bf z} and {\bf a} are both provided, libnucnet will assign the abundance
to the species as determined by the {\bf name}.

When there are multiple zones in the calculation, the zone data
part of the input data file has a form like the following
example:

\begin{verbatim}

<zone_data>

  <zone label1="x1" label2="y1" label3="z1">
    <mass_fractions>
      <nuclide>
        <z>0</z>
        <a>1</a>
        <x>0.5</x>
      </nuclide>
      <nuclide>
        <z>1</z>
        <a>1</a>
        <x>0.5</x>
      </nuclide>
    </mass_fractions>
  </zone>

  <zone label1="x2" label2="y2" label3="z2">
    <mass_fractions>
      <nuclide name="n">
        <z>0</z>
        <a>1</a>
        <x>0.5</x>
      </nuclide>
      <nuclide name="h1">
        <z>1</z>
        <a>1</a>
        <x>0.4</x>
      </nuclide>
      <nuclide name="he4">
        <z>2</z>
        <a>4</a>
        <x>0.1</x>
      </nuclide>
    </mass_fractions>
  </zone>

</zone_data>

\end{verbatim}

For each {\bf zone}, there is a set of {\bf mass\_fractions},
which consist of a sequence of nuclides ({\bf nuclide} tag).  As before,
the data for each nuclide is the name, atomic number, mass number, and mass
fraction of each species with non-zero abundance in the zone.
Also, the mass fractions within a given zone should all sum to unity.

libnucnet sets up a number of zones equal to the number of zones in
the input file.  The zones are labeled by the supplied label attributes.
A zone may have up to three labels, which are strings.
If no labels are provided, the first zone will be given labels such
that {\bf label1} = ``0", {\bf label2} = ``0", {\bf label3} = ``0",
the second zone will have
{\bf label1} = ``1", {\bf label2} = ``0", {\bf label3} = ``0",
the third zone will have
{\bf label1} = ``2", {\bf label2} = ``0", {\bf label3} = ``0", etc.
If some labels are provided for a zone but others are not, those labels
not provided are given the value ``0".  If identical labels are provided
for two zones, libnucnet error handling will be invoked.

As of version 0.3, it is possible to assign optional properties to a zone.
An optional property's value is stored as a string that is identified by
another string giving the property's name.  The user may also supply two
additional tags to identify the property.  The property can be assigned in the
input XML file.  An example would be:

\begin{verbatim}

<zone_data>

  <zone label1="1">
    <optional_properties>
       <property name="t9" tag1="0">10.</property>
       <property name="t9" tag1="1">9.</property>
       <property name="t9" tag1="2">8.</property>
       ...
       <property name="t9" tag1="20">0.1</property>
       <property name="time" tag1="0" tag2="seconds">0.</property>
       <property name="time" tag1="1" tag2="seconds">1.e-3</property>
       <property name="time" tag1="2" tag2="seconds">2.e-3</property>
       ...
       <property name="time" tag1="20">1.</property> 
   </optional_properties>
   <mass_fractions>
     ...
   </mass_fractions>
  </zone>

  ...

</zone_data>

\end{verbatim}

\section{Libnucnet}

Nuclear data, reaction data, and zone data can be combined into a single input
XML file, which can be read in with libnucnet API routines.  The combined
file looks like:

\begin{verbatim}

<libnucnet_input>

  <nuclear_network>

    <nuclear_data>
    ...
    </nuclear_data>

    <reaction_data>
    ...
    </reaction_data>

  </nuclear_network>

  <zone_data>
  ...
  </zone_data>

</libnucnet_input>

\end{verbatim}

This input can be read in with the libnucnet API routine
Libnucnet\_\_new\_\_from\_\_from\_\_xml().

\section{Using XInclude and XPointer}

The ability to include XML resources within other documents with
XInclude and XPointer is an extremely powerful feature of XML.
As of version 0.18 of libnucnet, it is possible to use XInclude.
For example, a full input libnucnet XML file could be constructed with
the following input XML document:

\begin{verbatim}
<?xml version="1.0" encoding="ISO-8859-1"?>

<libnucnet_input
  xmlns:xi="http://www.w3.org/2001/XInclude"
>

   <nuclear_network>

      <nuclear_data>

         <xi:include
           href="nuclear_data.xml"
           xpointer="xpointer(//nuclide)"
         />

      </nuclear_data>

      <reaction_data>

         <xi:include
           href="reaction_data1.xml"
           xpointer="xpointer(//reaction)"
         />

         <xi:include
           href="reaction_data2.xml"
           xpointer="xpointer(//reaction[reactant = 'ca40'])"
         />

      </reaction_data>

  </nuclear_network>

  <zone_data>

     <xi:include
       href="zone_data1.xml"
       xpointer="xpointer(//zone)"
     />

     <xi:include
       href="zone_data2.xml"
       xpointer="xpointer(//zone)"
     />

  </zone_data>

</libnucnet_input>

\end{verbatim}

When this XML file gets parsed by the libnucnet API routine
Libnucnet\_\_new\_from\_from\_xml(), the parser will include
all nuclide data from the XML file {\em nuclear\_data.xml}, all reaction
data from {\em reaction\_data1.xml} then reactions from
{\em reaction\_data2.xml} with $^{40}$Ca as a reactant.  Finally,
the zone data from {\em zone\_data1.xml} and then 
{\em zone\_data2.xml} gets parsed in.
Note that the {\em href} attribute could be any URI (uniform resource
identifier), such as a URL.  Also notice that it is possible to include,
for example, reaction data from several different reaction XML files.
If reaction data has already been read in, say, from {\em reaction\_data1.xml},
the data for that reaction will be updated with the new data from 
{\em reaction\_data2.xml}.  A new libnucnet example code (as of
version 0.18) demonstrates these capabilities.  Further examples to
demonstrate XInclude have been added in version 0.23.

In order to ensure that XML documents constructed with XInclude still
validate with our schemas, we originally (beginning in version 0.18)
did not fix up XInclude xml:base URIs in our validation routines.
We have decided that this is not restrictive enough in that it allows
inclusion of any data whereas our intention is that this should only be
possible for nuclides, reactions, and zones.  For this reason, we have
updated the libnucnet XML schemas to allow for xml:base attributes
in included nuclides, reactions, and zones.  libnucnet XML files with
properly included data will validate with libnucnet API routines.  They
will also validate with {\em xmllint} if one uses the flag
{\em {-}{-}xinclude}.

\begin{thebibliography}{3}
\expandafter\ifx\csname natexlab\endcsname\relax\def\natexlab#1{#1}\fi

\bibitem[{{Fowler} {et~al.}(1967){Fowler}, {Caughlan}, \&
  {Zimmerman}}]{1967ARA&A...5..525F}
{Fowler}, W.~A., {Caughlan}, G.~R., \& {Zimmerman}, B.~A. 1967, \araa, 5, 525,
  {Thermonuclear Reaction Rates}

\bibitem[{{Gupta} \& {Meyer}(2001)}]{2001PhRvC..64b5805G}
{Gupta}, S.~S. \& {Meyer}, B.~S. 2001, \prc, 64, 25805, {Internal equilibration
  of a nucleus with metastable states: $^{26}$Al as an example}

\bibitem[{{Rauscher} \& {Thielemann}(2000)}]{2000ADNDT..75....1R}
{Rauscher}, T. \& {Thielemann}, F. 2000, At. Data Nucl. Data Tables, 75, 1,
  {Astrophysical Reaction Rates From Statistical Model Calculations}

\end{thebibliography}

\end{document}
