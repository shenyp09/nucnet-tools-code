%///////////////////////////////////////////////////////////////////////////////
% <file type="public">
%
% <license>
%   See the README_module.xml file for this module for copyright and license
%   information.
% </license>
%   <description>
%     <abstract>
%       This is the Webnucleo Report for user-supplied rate functions to
%       libnucnet
%     </abstract>
%     <keywords>
%       libnucnet Module, webnucleo report, user, supplied, rate, function
%     </keywords>
%   </description>
%
%   <authors>
%     <current>
%       <author userid="mbradle" start_date="2009/09/03" />
%     </current>
%     <previous>
%     </previous>
%   </authors>
%
%   <compatibility>
%     TeX (Web2C 7.4.5) 3.14159 kpathsea version 3.4.5
%   </compatibility>
%
% </file>
%///////////////////////////////////////////////////////////////////////////////
%
% This is a sample LaTeX input file.  (Version of 9 April 1986)
%
% A '%' character causes TeX to ignore all remaining text on the line,
% and is used for comments like this one.

\documentclass{article}    % Specifies the document style.

\usepackage[dvips]{graphicx}
\usepackage{hyperref}

\def\apj{Astrophys. J}

                           % The preamble begins here.
\title{Webnucleo Technical Report: User-Supplied Rate Functions to libnucnet }

\author{Bradley S. Meyer}
%\date{December 12, 1984}   % Deleting this command produces today's date.

\begin{document}           % End of preamble and beginning of text.

\maketitle                 % Produces the title.


This technical report describes how to supply rate functions to libnucnet.

\section{User-Supplied Rate Functions}  \label{sec:user_rate_function}

The standard functions in libnucnet for calculating reaction rates are
single rates (numbers independent of the temperature, density, or any other
external parameter), rate tables (tables giving the rate vs. temperature,
which are interpolated), or non-smoker fit functions.  These standard
rate functions are described in the libnucnet input technical report.

In some cases users may wish to provide their own rate functions that
would be applied during reaction rate calculations.  As of version 0.5,
this is possible.  To do so, a user writes a
Libnucnet\_\_Reaction\_\_userRateFunction with the prototype
\begin{verbatim}
double
my_rate_function(
  Libnucnet__Reaction *p_reaction,
  double d_t9,
  void *p_user_data
);
\end{verbatim}
In this function, {\bf d\_t9} is the temperature in $10^9$ K and {\bf
p\_user\_data} is a pointer to a user-supplied data structure.  The user's
routine then returns the rate for the reaction.  The user must supply a
rate function for each different rate parameterization considered.

\section{Setting the Rate Function for a Reaction}

Once user rate functions are defined, they should be registered with
the reaction collection.  To do so, the user calls
Libnucnet\_\_Reac\_\_registerRateFunction().  For example, with the
{\bf my\_rate\_function} above, one would call
\begin{verbatim}
Libnucnet__Reac__registerRateFunction(
  p_my_reactions,
  "my rate function",
  (Libnucnet__Reaction__userRateFunction) my_rate_function
);
\end{verbatim} 
to register the function with the reaction collection
Libnucnet\_\_Reac *p\_my\_reactions.  In this example, the string
``my rate function'' is a key that allows libnucnet to retrieve and
apply the appropriate function to a reaction.

A reaction needs to have an associated function key.  If reaction data
are read in from XML, this is done automatically as the XML is parsed
via the {\bf key} attribute to the {\bf user\_rate} tag.
If a reaction is added separately, however, the user must set the
key directly.  The user does this by calling the
API routine Libnucnet\_\_Reaction\_\_setUserRateFunctionKey with the
prototype
\begin{verbatim}
void
Libnucnet__Reaction__setUserRateFunctionKey(
  Libnucnet__Reaction *p_reaction,
  const char *s_function_key
);
\end{verbatim}
In this function, s\_function\_key is a function key to a registered
user rate function. 

\section{Rate Data}

If a reaction has a user-supplied rate function, the user can set the
data for it by the Libnucnet\_\_Reaction\_\_updateUserRateFunctionProperty()
API routine.  Such data are typically reaction-rate fit parameters.
They are properties, which are strings
identified by a name and up to two optional tags.  The user may do this
directly or may set the data in the input xml file, as described in the
Webnucleo technical report on input xml to libnucnet.

Once the data for a reaction are set, they may be retrieved via the
Libnucnet\_\_Reaction\_\_getUserRateFunctionProperty() routine.  Here
the user supplies the reaction pointer and the strings for the name
of the property and the tags, if present.  The property is returned as a
string.  The properties may also be accessed by the API routine
Libnucnet\_\_Reaction\_\_iterateUserRateFunctionProperty().  Here the user
supplies a function with the prototype
\begin{verbatim}
void
my_iterate_function(
  const char *s_name,
  const char *s_tag1,
  const char *s_tag2,
  const char *s_value,
  void *p_data
);
\end{verbatim} 
Once this function is defined, the user then iterates over the properties
for the rate function for the reaction by calling, for example,
\begin{verbatim}
Libnucnet__Reaction__iterateUserRateFunctionProperties(
  p_reaction,
  s_name,
  s_tag1,
  s_tag2,
  (Libnucnet__Reaction__user_rate_property_iterate_function)
    my_iterate_function,
  p_data
);
\end{verbatim}
The iteration is over all properties that match s\_name, s\_tag1, and s\_tag2.
If any of these is NULL, the comparison is a match; thus, if all three are
NULL, the routine will iterate over all properties of the rate function
for the reaction.

Finally, the user may remove a property for a user rate function by calling
Libnucnet\_\_Reaction\_\_removeUserRateFunctionProperty().

\section{Extra Data}

A reaction rate is typically computed from fit data and from data characterizing
the physical conditions present at any point in the calculation.  The former
are what we have called ``rate data''.  These are data that apply to a
particular reaction.  The latter are extra, or ``user'', data that might
correspond to the density at a particular point in the calculation.
These user data are those supplied to the user's rate function, that is,
the data pointed to by {\bf p\_user\_data}, as described in \S
\ref{sec:user_rate_function}.

The extra data are typically associated with time-dependent quantities,
which, in turn, are associated with zones.  As of version 0.9, then,
to set these data, the user calls
Libnucnet\_\_Zone\_\_updateDataForUserRateFunction(), which has the
syntax:
\begin{verbatim}
void
Libnucnet__Zone__updateDataForUserRateFunction(
  Libnucnet__Zone *self,
  const char *s_function_key,
  void *p_data
);
\end{verbatim}
For example, to pass the density (stored as {\bf d\_density}) to a
user rate function registered with the key {\bf ``my rate function''}
in the zone {\bf p\_zone}, one would call
\begin{verbatim}
Libnucnet__Zone__updateDataForUserRateFunction(
  p_zone,
  "my rate function",
  &d_density
);
\end{verbatim}
Once this is done, any reaction rate computed from the function with
key ``my rate function'' would be computed in p\_zone
from the rate data for
the particular reaction and the current temperature $T_9$ and the
density d\_density set in the function above.

As of version 0.10, the user can assign a deallocator for the extra
data associated with a user-defined rate function.  The user writes a
routine with the prototype
\begin{verbatim}
void
Libnucnet__Reaction__user_rate_function_data_deallocator(
  void * p_data
);
\end{verbatim}
where {\bf p\_data} is a pointer to the user-defined data structure.  This
routine must free all memory associated with {\bf p\_data}, including the
structure itself.  Once this routine is written, the user then associates
it with a rate function by calling
Libnucnet\_\_Reac\_\_setUserRateFunctionDataDeallocator(), with prototype
\begin{verbatim}
int
Libnucnet__Reac__setUserRateFunctionDataDeallocator(
  Libnucnet__Reac * p_reac,
  const char * s_function_key,
  Libnucnet__Reaction__user_rate_function_data_deallocator
    pf_deallocator
);
\end{verbatim}
This routine associates the deallocator function {\bf pf\_deallocator} to
the user rate function defined by the key {\bf s\_function\_key} for the
reaction collection {\bf p\_reac}.  The routine returns 1 (true) if the
association succeeded and 0 (false) if not.
Once the deallocator is associated with
a rate function, a call to
Libnucnet\_\_Zone\_\_updateDataForUserRateFunction()
will employ the user's deallocation function to deallocate the memory for
the previously allocated extra data before adding the new data.

The current data in a zone for a user rate function can be retrieved
with the API function Libnucnet\_\_Zone\_\_getDataForUserRateFunction
with the prototype
\begin{verbatim}
void *
Libnucnet__Zone__getDataForUserRateFunction(
  Libnucnet__Zone * p_zone,
  const char * s_function_key
);
\end{verbatim}
This retrieves the current extra data for the rate function identified by
the rate function key s\_function\_key.  Examples in the libnucnet distribution
demonstrate how to apply extra data to a user function.

\end{document}
