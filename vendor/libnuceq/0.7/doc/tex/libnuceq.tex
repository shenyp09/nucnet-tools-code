%///////////////////////////////////////////////////////////////////////////////
% <file type="public">
%
% <license>
%   See the src/README.txt file for this module for copyright and license
%   information.
% </license>
%   <description>
%     <abstract>
%       This is the Webnucleo Report describing the computational details
%       behind the libnuceq module.
%     </abstract>
%     <keywords>
%       libnuceq, report
%     </keywords>
%   </description>
%
%   <authors>
%     <current>
%       <author userid="mbradle" start_date="2010/12/08" />
%     </current>
%     <previous>
%     </previous>
%   </authors>
%
%   <compatibility>
%     TeX (Web2C 7.4.5) 3.14159 kpathsea version 3.4.5
%   </compatibility>
%
% </file>
%///////////////////////////////////////////////////////////////////////////////
%
% This is a sample LaTeX input file.  (Version of 9 April 1986)
%
% A '%' character causes TeX to ignore all remaining text on the line,
% and is used for comments like this one.

\documentclass{article}    % Specifies the document style.

\usepackage[dvips]{graphicx}
\usepackage{hyperref}

\newcommand{\apj}{ApJ}

                           % The preamble begins here.
\title{Webnucleo Technical Report: Computational Details Behind libnuceq}  % Declares the document's title.

\author{Bradley S. Meyer, Tianhong Yu}

\begin{document}           % End of preamble and beginning of text.

\maketitle                 % Produces the title.


This technical report describes some details of calculations in the
libnuceq module.

\section{Introduction}

libnuceq is a library of C codes for computing nuclear statistical
equilibria relevant to nucleosynthesis.
It is built on top of libxml, the GNOME C xml toolkit, gsl,
the GNU Scientific library, and the Webnucleo.org modules wn\_matrix, libnucnet,
and libstatmech.
Users can compute abundances in arbitrary statistical equilibria, which
are specified by an XPath expression.
A well-documented API allows users to incorporate libnuceq into their
own codes, and examples in the libnuceq distribution demonstrate the API.

\section{Abundances}

The starting point for libnuceq calculations is the definition of $Y_i$, the
abundance per nucleon of nuclear species $i$.  This is defined as
\begin{equation}
Y_i = \frac{n_i}{\rho N_A}
\label{eq:yi}
\end{equation}
where $n_i$ is the number density of species $i$, $\rho$ is the mass density,
and $N_A$ is Avogadro's number.  We relate this to the chemical potential
of species $i$ (less its rest mass energy) by
\begin{equation}
Y_i = Y_{Qi} \exp\left\{\frac{\mu_i'}{kT} + f_{corr,i}\right\}.
\label{eq:yi_chem}
\end{equation}
In Eq. (\ref{eq:yi_chem}), $\mu_i'$ is the chemical potential of species
$i$ less the rest mass energy, $k$ is Boltzmann's constant, $T$ is the
temperature, and $f_{corr,i}$, a term that allows for deviations of the
abundance from this expression.
$Y_{Qi}$ is the quantum abundance per nucleon of species $i$.  It is the
abundance of species $i$ per nucleon that would obtain if there were one
particle of species $i$ in a cube with side equal to the thermal de Broglie
wavelength of species $i$.

In arbitrary nuclear statistical equilibria, there is a relation between
$\mu_i$, the chemical potential of species
$i$ and that of the neutrons $\mu_n$ and protons $\mu_p$:
\begin{equation}
\mu_i = \Lambda_i + Z_i \mu_p + \left( A_i - Z_i \right )\mu_n
\label{eq:nse}
\end{equation}
where $\Lambda_i$ is a quantity or function we call the {\em prefactor},
$Z_i$ is the atomic number of species $i$ and $A_i$ is its mass
number.  From Eqs. (\ref{eq:yi_chem}) and (\ref{eq:nse}), we find
\[
Y_i =
\]
\begin{equation}
 Y_{Qi} \exp\left\{\Lambda_i + Z_i \frac{\mu_p'}{kT} + \right(A_i - Z_i\left)
\frac{\mu_n'}{kT}
+ \frac{B_i}{kT} - \left( Z_i f_{corr,p} + \left( A_i - Z_i \right) f_{corr,n} -
f_{corr,i} \right)\right\}.
\label{eq:yi_final}
\end{equation}
In this equation, the binding energy $B_i$ is
\begin{equation}
B_i = Z_i m_pc^2 + \left( A_i - Z_i \right) m_nc^2 - m_ic^2,
\label{eq:binding}
\end{equation}
where $m_pc^2$, $m_nc^2$, and $m_ic^2$ are the rest mass energies of the
proton, neutron, and species $i$, respectively.

\section{Calculation of the Equilibria}

To solve for equilibria, we must satisfy certain abundance constraints.
The first, and the one always relevant under conditions in which
the nucleon rest mass energy is much greater than $kT$, is that nucleon
number is conserved:
\begin{equation}
\sum_i A_i Y_i = 1
\label{eq:a_i_y_i}
\end{equation}
where the sum runs on all nuclear species.  The remaining constraints
depend on the particular equilibrium considered.

\begin{enumerate}

\item \label{sec:wse}
The least constrained equilibrium is weak nuclear statistical equilibrium (WSE).
Here $\Lambda_i = 0$ and, if we assume zero chemical potential for all
neutrinos,
\begin{equation}
\mu_p + \mu_e = \mu_n
\label{eq:weak}
\end{equation}
This allows us to write
\begin{equation}
\mu_e' = \mu_n' - \mu_p' + \left( m_nc^2 - m_pc^2 - m_ec^2 \right)
\label{eq:mue'}
\end{equation}
Since we generally use atomic rather than nuclear masses, this becomes,
upon neglect of the binding energy of the electron in the H atom,
\begin{equation}
\frac{\mu_e'}{kT} =
\frac{\mu_n'}{kT} - \frac{\mu_p'}{kT} +
\frac{\left( m_nc^2 - m_{^1{\rm H}}c^2\right)}{kT}
\label{eq:muekT}
\end{equation}

The additional constraint on WSE is that the abundances must satisfy
charge neutrality:
\begin{equation}
\sum_i Z_i Y_i = Y_e
\label{eq:ye}
\end{equation}
where $Y_e$ is the net number of electrons per nucleon and the sum again
runs over nuclear species.  We note that for WSE,
$Y_e$ is not fixed; rather, it is a function of $T$ and $\mu_e / kT$.
The consequence is that we must simultaneously find the roots $\mu_n/kT$ and
$\mu_p/kT$ of the equations
\begin{equation}
f_1 = \sum_i A_i Y_i - 1
\label{eq:f1}
\end{equation}
and
\begin{equation}
f_2 = \sum_i Z_i Y_i - Y_e
\label{eq:f2}
\end{equation}
where the abundances are computed from Eq. (\ref{eq:yi_final}) with
$\Lambda_i = 0$ and $Y_e$ is computed from the electron chemical potential
in Eq. (\ref{eq:muekT}).

To solve Eqs. (\ref{eq:f1}) and (\ref{eq:f2}) with libnuceq, one first
creates an equilibrium with Libnuceq\_\_new().  The input to this routine
is a Libnucnet\_\_Nuc structure, which contains the relevant nuclear data.
Libnuceq\_\_new() returns a pointer to an equilibrium structure.
To solve for the WSE at a particular temperature and density, one calls
Libnuceq\_\_computeEquilibrium(), which takes as input the pointer to
the equilibrium and the temperature and density at which to compute the
abundances.

libnuceq solves Eqs. (\ref{eq:f1}) and (\ref{eq:f2}) as nested
one-dimensional root problems using the Brent solver gsl\_root\_fsolver\_brent
in the GNU Scientific Library.  We choose nested 1-d roots because it is
possible to bracket the roots and thus guarantee a solution.  Two-d methods
are faster but require a good initial guess.  We have opted for robustness
over speed.  Once the equilibrium has been found, the user may access the
results using API routines to get abundances for species or chemical
potentials for neutrons, protons, or electrons.

The default routine to compute $Y_e$ from $\mu_e'/kT$ is that for fully
relativistic, non-interacting electrons.  Should the user wish to employ
a different equation of state for the electrons, he or she can do so by
writing a Libstatmech\_\_Fermion\_\_Function and/or a
Libstatmech\_\_Fermion\_\_Integrand for the electron number density.
Once these are written,
the user then sets them with
Libnuceq\_\_updateUserElectronNumberDensity(), which has the
prototype
\begin{verbatim}
Libnuceq__updateUserElectronNumberDensity(
  Libnuceq *self,
  Libstatmech__Fermion__Function my_function,
  Libstatmech__Fermion__Integrand my_integrand,
  void *p_function_data,
  void *p_integrand_data
);
\end{verbatim}
Here $self$ is a pointer to the equilibrium that will use
the number density function
my\_function and the number density integrand
my\_integrand.  p\_function\_data and
p\_integrand\_data are pointers to user-defined extra data for the
function and the integrand.  The default calculation for the electron
number density is that for default libstatmech calculations
(non-interacting fully relativistic electrons).  If either the function
or integrand is set, it will be used in place of the default.
To restore the default, the user simply calls
Libnuceq\_\_updateUserElectronNumberDensity() with NULL for the
function, integrand, and data.

\item\label{sec:nse}
The next equilibrium is regular nuclear statistical equilibrium (NSE).
Here we assume weak reactions are slow so that the electron fraction does
not vary.  To compute NSE, then, one again solves Eqs. (\ref{eq:f1}) and
(\ref{eq:f2}) but for specified $Y_e$.  In libnuceq, one does this by
calling Libnuceq\_\_setYe() before computing the equilibrium.  This
routine takes as input the pointer to the equilibrium and the value of
the particular $Y_e$ at which to compute the NSE.  libnuceq then uses
the user-specified value of $Y_e$ in Eq. (\ref{eq:f2}).  To clear the
$Y_e$ constraint (and thereby restore the equilibrium to a WSE), one
calls the routine Libnuceq\_\_clearYe().  After the equilibrium has
been computed, the user may call API routines to get abundances and
chemical potentials.

\item\label{sec:qse}
A quasi-statistical equilibrium (QSE) is one for which there is an extra
constraint on some subset of nuclei.  The most common QSE occurs when
the number of heavy nuclei becomes a fixed number because the three-body
reactions that assemble them from $^4$He nuclei become slow
(e.g., \cite{1998ApJ...498..808M}).  The chemical potential for heavy
nuclei thus have a uniform shift from their usual NSE relation such that
$\Lambda_i = \mu_h / kT$ for all heavy nuclei $i$, where $\mu_h$ is
a chemical potential for the heavy nuclei as a whole.

To compute a QSE with libnuceq, one defines the relevant ``cluster'', that
is, the subset of nuclei with the same $\mu_h$.  These nuclei are in
equilibrium under the exchange of neutrons and protons.  To create
equilibrium cluster, the user first creates an equilibrium as in
\S \ref{sec:wse} and \ref{sec:nse}
and then creates a cluster with the routine
Libnuceq\_\_newCluster(), which has the prototype
\begin{verbatim}
Libnuceq__Cluster *
Libnuceq__newCluster(
  Libnuceq *self,
  const char *s_cluster_xpath
);
\end{verbatim}
Here $self$ is the equilibrium that will include the cluster and
s\_cluster\_xpath is an XPath expression that defines the cluster using
Libnucnet\_\_Nuc variables.  For example, one could define the QSE cluster
of all heavy nuclei to be carbon isotopes and above; thus, one could include
this cluster in the equilibrium pointed to by $p\_my\_equilibrium$ by
calling
\begin{verbatim}
p_cluster =
  Libnuceq__newCluster(
    p_my_equilibrium,
    "[z >= 6]"
  );
\end{verbatim}
This routine creates the cluster within $p\_my\_equilibrium$ and returns
a pointer to it.  The pointer to the cluster may subsequently be retrieved
by calling Libnuceq\_\_getCluster(), which gets the cluster by the
defining XPath expression.  For example, one would retrieve the heavy nuclei
cluster above by calling:
\begin{verbatim}
p_cluster =
  Libnuceq__getCluster(
    p_my_equilibrium,
    "[z >= 6]"
  );
\end{verbatim}

Once a cluster is defined, the user then sets the constraint on the
function by calling Libnuceq\_\_Cluster\_\_updateConstraint().  The
default constraint on a cluster is that the abundances of species
within the cluster sum up to a particular value.  Thus, for example,
if the sum of heavy nuclei is $Y_h = 0.01$, one would call
\begin{verbatim}
Libnuceq__Cluster__updateConstraint( p_cluster, 0.01 );
\end{verbatim}
When we solve for the equilibrium, libnuceq will simultaneously solve
Eqs. (\ref{eq:f1}), (\ref{eq:f2}), and
\begin{equation}
f_3 = \sum_{i\in C} Y_i - Y_h
\label{eq:f3}
\end{equation}
for the roots $\mu_n'/kT$, $\mu_p'/kT$ and $\mu_h/kT$.  In Eq. (\ref{eq:f3})
the sum extends only over species contained in cluster $C$. Again, once the
equilibrium has been computed abundances and chemical potentials, including
the chemical potential of the cluster, may be retrieved with API routines.

It is possible to define multiple clusters--one simply calls
Libnuceq\_\_newCluster() for each cluster and sets the constraint on each.
It is important to note that clusters should not overlap, that is, a
species should not belong to more than one cluster.  Also, clusters should
not include neutrons and protons.

As of version 0.5, it is possible to copy clusters from one equilibrium
to another if they share the same parent nuclide collection.  This is
advantageous, for example, if one wants to compare two equilibria.
The API routine that does this is Libnuceq\_\_copy\_clusters().  While one
can simply create new clusters in the second equilibrium, this can
be computationally expensive due to XPath evaluations.
It is thus convenient to be able simply to copy over already
existing clusters from the first equilibrium. 

\item\label{sec:rnse}
A restricted NSE is one for which the NSE extends only over a subset of nuclei.
The cluster (subset) $C$
that does not include the neutrons and protons, then, has
a $\Lambda_i = A_i \mu_r / kT$ and a cluster constraint equation
\begin{equation}
f_3 = \sum_{i\in C} A_i Y_i - X_r
\label{eq:f3_r}
\end{equation}
where $X_r$ is the mass fraction of species contained within $C$.
Here both the prefactor $\Lambda_i$ and the constraint function
Eq. (\ref{eq:f3_r}) differ from the defaults.  To specify a different
prefactor, the user writes a Libnuceq\_\_Cluster\_\_prefactorFunction,
which has the prototype
\begin{verbatim}
double
my_prefactor_function(
  Libnuceq__Cluster *p_cluster,
  Libnuceq__Species *p_species,
  void *p_my_prefactor_data
);
\end{verbatim}
where {\em p\_cluster} is the cluster to which the prefactor function is
applied, {p\_species} is the particular nuclear species $i$, and
$p\_my\_prefactor\_data$
is a pointer to user-defined extra data to the function.  The user can
then write a Libnuceq\_\_Cluster\_\_constraint\_function,
which has the prototype
\begin{verbatim}
double
my_constraint_function(
  Libnuceq__Species *p_species,
  void *p_my_constraint_data
);
\end{verbatim}
The names my\_prefactor\_function and my\_constraint\_function
are for illustration--the user may choose different names that make
sense to him or her.

Once the prefactor and constraint functions are written, the user then
sets them for an equilibrium with the API routines
Libnuceq\_\_Cluster\_\_updatePrefactorFunction()
and
Libnuceq\_\_Cluster\_\_updateConstraintFunction().  For the examples above,
the calls would be, for a cluster $p\_my\_cluster$,
\begin{verbatim}
Libnuceq__Cluster__updatePrefactorFunction(
  p_my_cluster,
  Libnuceq__Cluster__prefactorFunction my_prefactor_function,
  p_my_prefactor_data
);
\end{verbatim}
and
\begin{verbatim}
Libnuceq__Cluster__updateConstraintFunction(
  p_my_cluster,
  Libnuceq__Cluster__constraint_function my_constraint_function,
  p_my_constraint_data
);
\end{verbatim}
Now when the equilibrium is computed, libnuceq calculates the prefactor
and constraint from the user's functions.

\end{enumerate}

Examples in the libnuceq distribution illustrate how to compute equilibria
and to set constraints.

\section{Setting Correction Factors}

libnuceq allows users to correct abundances for deviations ($f_{corr,i}$)
away from the simple relation that $Y_i = Y_{Qi}\exp(\mu'/kT)$.
The correction is computed by a user-defined
Libnucnet\_\_Species\_\_nseCorrectionFactorFunction.  The prototype
for this function is (see libnucnet documentation for further details)
\begin{verbatim}
double
my_corr_function(
  Libnucnet__Species *p_species,
  double d_t9,
  double d_rho,
  double d_ye,
  void *p_nse_corr_data
);
\end{verbatim}
Here {\em d\_t9} is the temperature in $10^9$ K, {\em d\_rho}
is the mass density
in g/cc, {\em d\_ye} is the electron fraction $Y_e$, and
{\em p\_nse\_corr\_data}
is a pointer to user-defined extra data for the function.  The user
writes this function and then sets it for a particular equilibrium with
Libnuceq\_\_setNseCorrectionFactorFunction(), which has the prototype
\begin{verbatim}
void
Libnuceq__setNseCorrectionFactorFunction(
  Libnuceq *p_my_equilibrium,
  Libnucnet__Species__nseCorrectionFactorFunction my_corr_function,
  void *p_nse_corr_data
);
\end{verbatim}
Now when libnuceq computes the equilibrium, it will compute $f_{corr,i}$
for each species from the user's correction factor function (in the
above example, my\_corr\_function).  Note that libnuceq will call
the correction factor function with the temperature and density set
by the user in calling Libnuceq\_\_computeEquilibrium() and the
current $Y_e$ for the equilibrium (whether set by the user with
Libnuceq\_\_setYe() or computed from $\mu_e'/kT$).  To restore the default
(no correction factor function), the user calls
Libnuceq\_\_clearNseCorrectionFactorFunction().  Examples in the
libnuceq distribution illustrate how to do this.

\section{Setting Weak Statistical Equilibrium Correction Factors}

As of version 0.3 of libnuceq, it is possible to add a correction to the
relation among neutrons, protons, and electrons in WSE.
Here Eq. (\ref{eq:weak}) is modified to read
\begin{equation}
\frac{\mu_p}{kT} + \frac{\mu_e}{kT} = \frac{\mu_n}{kT} + f_{wse,corr}.
\label{eq:weak_modified}
\end{equation}
$f_{wse,corr}$ is the WSE correction function.  For example, if the
electron neutrino chemical potential is non zero, then
\begin{equation}
f_{wse,corr} = \frac{\mu_{\nu_e}}{kT}
\label{eq:mu_nue}
\end{equation}

To add a correction, the user writes a routine with the prototype
\begin{verbatim}
double
my_wse_correction_function(
  Libnuceq * p_my_equilibrium,
  void * p_my_wse_corr_data
);
\end{verbatim}
Here $p\_my\_equilibrium$ is a pointer to the equilibrium to which the
WSE correction will be applied and $p\_my\_data$ is a pointer to a
user-supplied data structure containing extra data for the correction
function.  The user's routine must compute the correction $f_{wse,corr}$
for the supplied input.

Once the user has written such a correction function, the user applies
it to an equilibrium with the libnuceq API routine
Libnuceq\_\_updateWseCorrectionFunction(), which has the prototype
\begin{verbatim}
void
Libnuceq__updateWseCorrectionFunction(
  Libnuceq * p_my_equilibrium,
  Libnuceq__wseCorrectionFunction my_wse_correction_function,
  void * p_my_wse_corr_data
);
\end{verbatim}
The calculation of the WSE for $p\_my\_equilibrium$ will then use
$my\_wse\_correction\_function$ with $p\_my\_wse\_corr\_data$.  To
clear the WSE correction function for an equilibrium, the user calls
$Libnuceq\_\_updateWseCorrectionFunction()$ with the both
the correction function and user data set to NULL.

Documentation and examples in the libnuceq distribution provide further
details.

\begin{thebibliography}{1}

\bibitem{1998ApJ...498..808M}
{\sc B.~S. {Meyer}, T.~D. {Krishnan}, and D.~D. {Clayton}}, {\em Theory of
  quasi-equilibrium nucleosynthesis and applications to matter expanding from
  high temperature and density}, \apj, 498 (1998), pp.~808--830.

\end{thebibliography}


\end{document}
