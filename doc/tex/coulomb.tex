\documentclass{article}

\usepackage{hyperref}
\hypersetup{colorlinks=true}

\newcommand{\mnras}{Mon. Not. Royal. Astron. Soc.}
\newcommand{\pra}{Phys. Rev. A}

\begin{document}

\title{Default NSE Corrections in NucNet Tools}
\author{Bradley S. Meyer}

\maketitle

This document presents some notes on the default Coulomb corrections used
in \href{http://nucnet-tools.sourceforge.net}{NucNet Tools}.
We use the expression for the Helmholtz free energy in
the one-component plasma (OCP) given in \cite{1980PhRvA..21.2087S}:
\begin{equation}
\frac{F(\Gamma)}{NkT} = g_1(\Gamma) =
a \Gamma + 4 b \Gamma^{1/4} - 4 c \Gamma^{-1/4} + d\ln(\Gamma) -
\left[ a + 4 (b - c) \right] + \frac{F(1)}{kT}
\label{eq:f}
\end{equation}
where $k$ is Boltzmann's constant and $T$ is the temperature.
Here, the Coulomb coupling parameter $\Gamma$ is given by
\begin{equation}
\Gamma = Z^{5/3} \Gamma_e
\end{equation}
where $Z$ is the ion charge and $\Gamma_e$ is given by
\begin{equation}
\Gamma_e = \left(\frac{e^2}{a_e k T}\right)
\end{equation}
with the electron cloud radius $a_e$ given in terms of the electron number
density $n_e$ as
\begin{equation}
a_e = \left( \frac{3}{4 \pi n_e} \right)^{1/3}
\end{equation}
As suggested by \cite{1999MNRAS.307..984B}, when $\Gamma < 1$ we take
\begin{equation}
\frac{F(\Gamma)}{NkT} = g_2(\Gamma) = -\frac{1}{\sqrt{3}} \Gamma^{3/2} +
\frac{\beta}{\gamma} \Gamma^\gamma
\label{eq:f1}
\end{equation}
The constants $\beta$ and $\gamma$ are determined by
matching the expressions in Eqs (\ref{eq:f}) and (\ref{eq:f1}) and
their first derivatives at $\Gamma = 1$.

It is possible to compute thermodynamic quantities from Eq. (\ref{eq:f})
or Eq. (\ref{eq:f1}).  The chemical potential $\mu$ is given by
\begin{equation}
\mu = \left(\frac{\partial F}{\partial N}\right)_{T,V}
\end{equation}
If we do not change $n_e$ in the variation, then we simply find
\begin{equation}
\frac{\mu}{kT} = g(\Gamma)
\end{equation}
where we choose $g = g_1$ for $\Gamma > 1$ and $g = g_2$ for $\Gamma < 1$.

The entropy $S$ is given by
\begin{equation}
S = -\left(\frac{\partial F}{\partial T}\right)_{V,N}
\end{equation}
We first note that
\begin{equation}
\left(\frac{\partial \Gamma}{\partial T}\right)_{V,N} = -\frac{\Gamma}{T}
\end{equation}
With this result, we note that
\begin{equation}
\frac{S}{Nk} = -g + \Gamma \frac{\partial g}{\partial \Gamma}
\end{equation}
where $g$ is either $g_1$ or $g_2$.
For $\Gamma > 1$, $g = g_1$, so
\begin{equation}
\Gamma \frac{\partial g}{\partial \Gamma} =
a \Gamma + b \Gamma^{1/4} + c \Gamma^{-1/4} + d
\label{eq:gdgdl}
\end{equation}
From this, we find
\begin{equation}
\frac{S}{Nk} = -3 b \Gamma^{1/4} + 5 c \Gamma^{-1/4} + d - d\ln(\Gamma) + o
\end{equation}
where
\begin{equation}
o = a + 4 (b - c) - \frac{F(1)}{kT}
\end{equation}
For $\Gamma < 1$, $g = g_2$, so
\begin{equation}
\Gamma \frac{\partial g}{\partial \Gamma} =
-\frac{\sqrt{3}}{2} \Gamma^{3/2} + \beta \Gamma^\gamma
\label{eq:gdgdl2}
\end{equation}
In this case, the entropy for $\Gamma < 1$ is
\begin{equation}
\frac{S}{Nk} = -\frac{1}{2\sqrt{3}} \Gamma^{3/2} +
\beta \left( 1 - \frac{1}{\gamma}\right) \Gamma^\gamma
\end{equation}

The energy $U$ is given by
\begin{equation}
\frac{U}{NkT} = \frac{F + TS}{NkT} = \frac{F}{NkT} + \frac{S}{Nk} =
\Gamma \frac{\partial g}{\partial \Gamma}
\end{equation}
For $\Gamma > 1$, we find from Eq. (\ref{eq:gdgdl})
\begin{equation}
\frac{U}{NkT} = 
a \Gamma + b \Gamma^{1/4} + c \Gamma^{-1/4} + d
\end{equation}
For $\Gamma < 1$, we find from Eq. (\ref{eq:gdgdl2})
\begin{equation}
\frac{U}{NkT} = 
-\frac{\sqrt{3}}{2} \Gamma^{3/2} + \beta \Gamma^\gamma
\end{equation}

The pressure $P$ is given by
\begin{equation}
P = -\left( \frac{\partial F}{\partial V}\right)_{T,N}
\end{equation}
From the expression for $\Gamma$, we find
\begin{equation}
\left(\frac{\partial \Gamma}{\partial V}\right)_{T,N} = -\frac{\Gamma}{3V}
\end{equation}
With this,
\begin{equation}
P = \frac{NkT}{3V} \Gamma \frac{\partial g}{\partial \Gamma}
= \frac{1}{3}\frac{U}{V}
\end{equation}

For the default NSE correction treatment in NucNet Tools, 
the constants $a$, $b$, $c$, and $d$ are taken from \cite{1987PhRvA..36.5451O}.
The value of $F(1) / NkT = -0.420$, as suggested by \cite{1980PhRvA..21.2087S}.

When more than one ion species is present (a multi-component plasma),
we use the usual ``additive approximation'' in which each species
is computed as if it were a OCP and then the contributions of each ion are
weighted by the ion's abundance.  In this case, we consider the abundance
per nucleon of species $i$ as $Y_i = N_i / N$, where $N$ is the total
number of nucleons in the system.  The Coulomb entropy per nucleon (in
units of $k$) for species $i$ is then
\begin{equation}
s_i = Y_i \left( -g + \Gamma \frac{\partial g}{\partial \Gamma} \right)
\end{equation}
Similarly, the energy per nucleon for species $i$ is
\begin{equation}
\frac{u_i}{kT} = Y_i \Gamma \frac{\partial g}{\partial \Gamma}
\end{equation}
The partial pressure for species $i$ is
\begin{equation}
P_i = \frac{1}{3} n u_i
\end{equation}
where $n$ is the number density of total nucleons.

\bibliographystyle{plain}

\begin{thebibliography}{1}

\bibitem{1999MNRAS.307..984B}
E.~{Bravo} and D.~{Garc{\'{\i}}a-Senz}.
\newblock {Coulomb corrections to the equation of state of nuclear statistical
  equilibrium matter: implications for SNIa nucleosynthesis and the
  accretion-induced collapse of white dwarfs}.
\newblock {\em \mnras}, 307:984--992, August 1999.

\bibitem{1987PhRvA..36.5451O}
S.~{Ogata} and S.~{Ichimaru}.
\newblock {Critical examination of N dependence in the Monte Carlo calculations
  for a classical one-component plasma}.
\newblock {\em \pra}, 36:5451--5454, December 1987.

\bibitem{1980PhRvA..21.2087S}
W.~L. {Slattery}, G.~D. {Doolen}, and H.~E. {Dewitt}.
\newblock {Improved equation of state for the classical one-component plasma}.
\newblock {\em \pra}, 21:2087--2095, June 1980.

\end{thebibliography}

\end{document}
